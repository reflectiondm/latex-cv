%%%%%%%%%%%%%%%%%%%%%%%%%%%%%%%%%%%%%%%%%
% Developer CV
% LaTeX Template
% Version 1.0 (28/1/19)
%
% This template originates from:
% http://www.LaTeXTemplates.com
%
% Authors:
% Jan Vorisek (jan@vorisek.me)
% Based on a template by Jan Küster (info@jankuester.com)
% Modified for LaTeX Templates by Vel (vel@LaTeXTemplates.com)
%
% License:
% The MIT License (see included LICENSE file)
%
%%%%%%%%%%%%%%%%%%%%%%%%%%%%%%%%%%%%%%%%%

%----------------------------------------------------------------------------------------
%	PACKAGES AND OTHER DOCUMENT CONFIGURATIONS
%----------------------------------------------------------------------------------------

\documentclass[9pt]{developercv} % Default font size, values from 8-12pt are recommended

%----------------------------------------------------------------------------------------

\begin{document}

%----------------------------------------------------------------------------------------
%	TITLE AND CONTACT INFORMATION
%----------------------------------------------------------------------------------------

\begin{minipage}[t]{0.45\textwidth} % 45% of the page width for name
	\vspace{-\baselineskip} % Required for vertically aligning minipages
	
	% If your name is very short, use just one of the lines below
	% If your name is very long, reduce the font size or make the minipage wider and reduce the others proportionately
	\colorbox{black}{{\HUGE\textcolor{white}{\textbf{\MakeUppercase{Andrei}}}}} % First name
	
	\colorbox{black}{{\HUGE\textcolor{white}{\textbf{\MakeUppercase{Zubov}}}}} % Last name
	
	\vspace{6pt}
	
	{\huge Software Engineer} % Career or current job title
\end{minipage}
\begin{minipage}[t]{0.275\textwidth} % 27.5% of the page width for the first row of icons
	\vspace{-\baselineskip} % Required for vertically aligning minipages
	
	% The first parameter is the FontAwesome icon name, the second is the box size and the third is the text
	% Other icons can be found by referring to fontawesome.pdf (supplied with the template) and using the word after \fa in the command for the icon you want
	\icon{MapMarker}{12}{Munich, Germany}\\
	\icon{Phone}{12}{+49 1522 564 6009}\\
	\icon{At}{12}{\href{mailto:reflection.dm@gmail.com}{reflection.dm@gmail.com}}\\	
\end{minipage}
\begin{minipage}[t]{0.275\textwidth} % 27.5% of the page width for the second row of icons
	\vspace{-\baselineskip} % Required for vertically aligning minipages
	
	% The first parameter is the FontAwesome icon name, the second is the box size and the third is the text
	% Other icons can be found by referring to fontawesome.pdf (supplied with the template) and using the word after \fa in the command for the icon you want
	%\icon{Globe}{12}{\href{https://alyx.vance.me}{alyx.vance.me}}\\
	\icon{Github}{12}{\href{https://github.com/reflectiondm}{github.com/reflectiondm}}\\
	\icon{Linkedin}{12}{\href{https://linkedin.com/in/andreizubov}{/in/andreizubov}}\\
\end{minipage}

\vspace{0.5cm}

%----------------------------------------------------------------------------------------
%	INTRODUCTION, SKILLS AND TECHNOLOGIES
%----------------------------------------------------------------------------------------

\cvsect{Who Am I?}

\begin{minipage}[t]{0.4\textwidth} % 40% of the page width for the introduction text
	\vspace{-\baselineskip} % Required for vertically aligning minipages
	I have been working in the software development industry for more than 10 years and I
	am still passionate and excited about it. I believe that great software is built
	only when people work together, share their ideas and insights and push the whole
	community forward. I like coding, making ideas a reality, and I enjoy working with
	people, mentoring and inspiring them, organizing community activities, and pushing
	the limits of what people think is possible.
\end{minipage}
\hfill % Whitespace between
\begin{minipage}[t]{0.5\textwidth} % 50% of the page for the skills bar chart
	\vspace{-\baselineskip} % Required for vertically aligning minipages
	\begin{barchart}{5.5}
		\baritem{JS/TS}{90}
		\baritem{HTML}{80}
		\baritem{CSS}{80}
		\baritem{React}{90}
		\baritem{AWS}{40}
		\baritem{Kubernetes}{50}
		\baritem{Node JS}{80}
	\end{barchart}
\end{minipage}
\\ \\ 
\begin{minipage}[t]{0.3\textwidth}
	\vspace{-\baselineskip} % Required for vertically aligning minipages

	\cvsect{Development Practices}
	
	Agile, SCRUM, Kanban, CI/CD, \\Microfrontend architecture,\\ Microservices architecture,\\ Design patterns, TDD, BDD,\\ Refactoring practices,\\ Domain Driven Design, \\Code review,\\ Pair programming
\end{minipage}
\hfill
\begin{minipage}[t]{0.3\textwidth}
	\vspace{-\baselineskip} % Required for vertically aligning minipages
	
	\cvsect{WEB}
	
	REST architecture, Single Page Applications, Node.js, GraphQL, Jest, Webdriver
\end{minipage}
\hfill
\begin{minipage}[t]{0.3\textwidth}
	\vspace{-\baselineskip} % Required for vertically aligning minipages
	
	\cvsect{Tools}
	
	Git,  Docker, Kubernetes, NewRelic, Elastic stack, AWS
\end{minipage}
\\ \\ 

%----------------------------------------------------------------------------------------
%	EXPERIENCE
%----------------------------------------------------------------------------------------

\cvsect{Experience}

\begin{entrylist}
	\entry
		{2019 -- present\\}
		{Engineering Manager}
		{{\href{https://www.westwingnow.de/}{Westwing}}}
		{I joined Westwing as a Software Developer for a brand new team working on re-platforming the {\em {\href{https://www.westwingnow.de/}{WestwingNow}}} 
		frontend solution from PHP to serverside rendered react
		application. After a few months in the team, I've assumed the responsibilities of a people manager together with being responsible for the technical 
		design of our application. My goal was to make our team performant and predictable, ensuring timely delivery of features necessary to conclude our 
		re-platforming project.  I focused on enabling my team to constantly improve and hone their skills, providing guidance, and giving them enough freedom 
		to share their ideas. All that allowed us to get back on track with the project and deliver a high-quality product. Apart from doing team-related 
		activities I've also kickstarted company-wide knowledge sharing sessions in the form of lightning talks. As our re-platforming project has proven to 
		be successful and in the light of a very successful 2020, the company decided to proceed with further expanding our new tech platform to other areas of 
		the website. To speed up the process it was decided to start new teams responsible for other website domains, and a special growth team that would aim to improve business KPIs on existing pages.
		My top priority was keeping the engineering culture that we've managed to build during this considerable headcount growth phase as well as pushing the 
		frontend system architecture in a direction that would allow multiple teams to work at their own pace without impacting each other deliverables.
		\\ \texttt{node.js}\slashsep\texttt{React}\slashsep\texttt{AWS}\slashsep\texttt{K8S}\slashsep\texttt{Microfrontends}}
	\entry
		{2016 -- 2018}
		{Full stack developer}
		{{\href{https://www.check24.de/}{Check24}}}
		{I was a part of a team building an {\em {\href{https://www.check24.de/unfallversicherung/}{Unfallversicherung}}} (accident insurance) part of Check24 web portal. 
		It was a rich single page application built with knockout.js and a
		backend build by micro services architecture approach combining .net and node.js for tasks where they fit best. Another web application we developed was 
		a customer area, where the client can access previously created contracts or offers. It was a single page application built with react and MobX. I 
		contributed to our application on all levels from front-end part, down to back end and internal development and analytical tools. I introduced and 
		set up elastic stack installation for monitoring and analysis of our microservices landscape. I also work with junior developers and working students 
		providing guidance and making sure they learn our code-style, best practices and tools we use and helping them quickly get up to speed.
		\\ \texttt{React}\slashsep\texttt{Knockout.js}\slashsep\texttt{Mongo DB}\slashsep\texttt{ASP.NET WebAPI}\texttt{.NET}\slashsep\texttt{Elastic Stack}}
	\entry
		{2013 -- 2016}
		{Senior Application Developer}
		{Deutsche Bank}
		{I worked on the User Interface for an order management system used by sales traders in the front office. This was a click-once 
		distributed plug-in based WPF application. The codebase had a high level of unit test coverage, and used the inversion of control 
		pattern extensively. I was responsible for implementing new business features from server interaction to controls and 
		visualizations; designing new subsystems of our application and refactoring the existing ones. I initiated the transition 
		process of the codebase from SVN to git repository using Atlassian Stash as a main repo. This helped to achieve 100 percent 
		code review coverage for committed code.
		\\ \texttt{.NET}\slashsep\texttt{WPF}\slashsep\texttt{Autofac}}
	\entry
		{2012 -- 2013}
		{Senior Software Developer}
		{Positive Technologies}
		{I worked on a kanban board web plugin for Team Foundation Server 2012, CI Pipelines and desktop application for a vulnerability scanner.
		\\ \texttt{.NET}\slashsep\texttt{WPF}\slashsep\texttt{Castle Windsor}}
	\entry
		{2011 -- 2012}
		{.NET Developer}
		{Ols-Komplekt}
		{I worked on a multi-tier application in the domain of 'working condition analysis'. The technologies used throughout the application were WPF, WCF, EntityFramework, Enterprise Library, Prism, Workflow Services. We had two distributed teams, working with SCRUM methodology, and using development practices such as TDD, code reviews and continuous integration. I was a scrum master in the team for several months.
		\\ \texttt{.NET}\slashsep\texttt{WPF}\slashsep\texttt{Entity Framework}}
	\entry
		{2010 -- 2011}
		{.NET Developer}
		{Medline Soft}
		{I developed an information system for medical institutions.
		\\ \texttt{WinForms},\slashsep\texttt{DevExpress},\slashsep\texttt{MSSQL},\slashsep\texttt{TFS}}
	\entry
		{2008 -- 2010}
		{.NET Developer}
		{Namip OR}
		{I worked on an application for Motorola mobile barcode scanners. The application was built using the .net compact framework for Windows CE, and consisted of a Winforms based client, and a web-service with an Oracle database for long term storage.}
\end{entrylist}

%----------------------------------------------------------------------------------------
%	EDUCATION
%----------------------------------------------------------------------------------------

\cvsect{Education}

\begin{entrylist}
	\entry
		{2004 -- 2010}
		{Specialist (combined BSc and Master’s Degree)}
		{Bauman Moscow State Technical University}
		{Plasma Physics and Technology}
\end{entrylist}

%----------------------------------------------------------------------------------------
%	ADDITIONAL INFORMATION
%----------------------------------------------------------------------------------------

\begin{minipage}[t]{0.3\textwidth}
	\vspace{-\baselineskip} % Required for vertically aligning minipages

	\cvsect{Languages}
	
	\textbf{Russian} - native\\
	\textbf{English} - fluent\\
	\textbf{German} - Intermediate
\end{minipage}
\hfill
\begin{minipage}[t]{0.3\textwidth}
	\vspace{-\baselineskip} % Required for vertically aligning minipages
	
	\cvsect{Hobbies}
	
	I love snowboarding, climbing, playing guitar and mixing cocktails
\end{minipage}
\hfill
\begin{minipage}[t]{0.3\textwidth}
	\vspace{-\baselineskip} % Required for vertically aligning minipages
	
	\cvsect{What drives me}
	
	I want to make the world a better place. Be it code, product, or team environment -
	I want to leave it in a better state than I found it in.
\end{minipage}

%----------------------------------------------------------------------------------------

\end{document}
